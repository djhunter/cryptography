% Options for packages loaded elsewhere
\PassOptionsToPackage{unicode}{hyperref}
\PassOptionsToPackage{hyphens}{url}
%
\documentclass[
  twoside]{article}
\usepackage{amsmath,amssymb}
\usepackage{lmodern}
\usepackage{iftex}
\ifPDFTeX
  \usepackage[T1]{fontenc}
  \usepackage[utf8]{inputenc}
  \usepackage{textcomp} % provide euro and other symbols
\else % if luatex or xetex
  \usepackage{unicode-math}
  \defaultfontfeatures{Scale=MatchLowercase}
  \defaultfontfeatures[\rmfamily]{Ligatures=TeX,Scale=1}
\fi
% Use upquote if available, for straight quotes in verbatim environments
\IfFileExists{upquote.sty}{\usepackage{upquote}}{}
\IfFileExists{microtype.sty}{% use microtype if available
  \usepackage[]{microtype}
  \UseMicrotypeSet[protrusion]{basicmath} % disable protrusion for tt fonts
}{}
\makeatletter
\@ifundefined{KOMAClassName}{% if non-KOMA class
  \IfFileExists{parskip.sty}{%
    \usepackage{parskip}
  }{% else
    \setlength{\parindent}{0pt}
    \setlength{\parskip}{6pt plus 2pt minus 1pt}}
}{% if KOMA class
  \KOMAoptions{parskip=half}}
\makeatother
\usepackage{xcolor}
\usepackage[left=0.5in, right=0.5in, top=0.8in, bottom=0.4in]{geometry}
\usepackage{graphicx}
\makeatletter
\def\maxwidth{\ifdim\Gin@nat@width>\linewidth\linewidth\else\Gin@nat@width\fi}
\def\maxheight{\ifdim\Gin@nat@height>\textheight\textheight\else\Gin@nat@height\fi}
\makeatother
% Scale images if necessary, so that they will not overflow the page
% margins by default, and it is still possible to overwrite the defaults
% using explicit options in \includegraphics[width, height, ...]{}
\setkeys{Gin}{width=\maxwidth,height=\maxheight,keepaspectratio}
% Set default figure placement to htbp
\makeatletter
\def\fps@figure{htbp}
\makeatother
\setlength{\emergencystretch}{3em} % prevent overfull lines
\providecommand{\tightlist}{%
  \setlength{\itemsep}{0pt}\setlength{\parskip}{0pt}}
\setcounter{secnumdepth}{-\maxdimen} % remove section numbering
\usepackage[charter]{mathdesign}
\usepackage{fancyhdr}
\pagestyle{fancy}
\lhead{MA/CS-124, Hunter, Fall 2022}
\rhead{Syllabus, Page \thepage}
\cfoot{}
\setlength{\headsep}{0.2in}

\usepackage{enumitem}

\setenumerate{leftmargin=*}

\newcommand{\vect}[1]{\mathbf{#1}} % vectors in bold face

%\newcommand{\vect}[1]{\mathbf{#1}\,} % vectors in bold face (need thin
                                     % space after)
%\newcommand{\vect}[1]{\vec{#1}} %vectors as arrows

\providecommand{\norm}[1]{\left\lvert#1\right\rvert} % norms as single lines
%\providecommand{\norm}[1]{\left\lVert#1\right\rVert} % norms as double lines

% bold or blackboard bold!?
\newcommand{\NN}{\mathbb{N}}
%\newcommand{\RR}{\mathbf{R}}
\newcommand{\RR}{\mathbb{R}}

% some useful abbreviations
\newcommand{\vx}{\vect{x}}
\newcommand{\vy}{\vect{y}}
\newcommand{\vz}{\vect{z}}
\newcommand{\vu}{\vect{u}}
\newcommand{\vv}{\vect{v}}
\newcommand{\vw}{\vect{w}}
\newcommand{\va}{\vect{a}}
\newcommand{\vb}{\vect{b}}
\newcommand{\vc}{\vect{c}}
\newcommand{\ve}{\vect{e}}
\newcommand{\vf}{\vect{f}}
\newcommand{\vF}{\vect{F}}
\newcommand{\vg}{\vect{g}}
\newcommand{\vh}{\vect{h}}
\newcommand{\vl}{\vect{l}}
\newcommand{\vm}{\vect{m}}
\newcommand{\vn}{\vect{n}}
\newcommand{\vp}{\vect{p}}
\newcommand{\vr}{\vect{r}}
\newcommand{\vs}{\vect{s}}
\newcommand{\vi}{\vect{i}}
\newcommand{\vj}{\vect{j}}
\newcommand{\vk}{\vect{k}}
\newcommand{\vzero}{\vect{0}}
% lower-case greek letters are handled differently: poor man's bold macro
%\newcommand{\vphi}{\pmb{\phi}}

\newcommand{\malename}{Westley} % recurring male name
\newcommand{\femalename}{Buttercup} % recurring female name
\usepackage{booktabs}
\usepackage{longtable}
\usepackage{array}
\usepackage{multirow}
\usepackage{wrapfig}
\usepackage{float}
\usepackage{colortbl}
\usepackage{pdflscape}
\usepackage{tabu}
\usepackage{threeparttable}
\usepackage{threeparttablex}
\usepackage[normalem]{ulem}
\usepackage{makecell}
\usepackage{xcolor}
\ifLuaTeX
  \usepackage{selnolig}  % disable illegal ligatures
\fi
\IfFileExists{bookmark.sty}{\usepackage{bookmark}}{\usepackage{hyperref}}
\IfFileExists{xurl.sty}{\usepackage{xurl}}{} % add URL line breaks if available
\urlstyle{same} % disable monospaced font for URLs
\hypersetup{
  hidelinks,
  pdfcreator={LaTeX via pandoc}}

\author{}
\date{\vspace{-2.5em}}

\begin{document}

\hypertarget{codes-and-encryption-macs-124-westmont-college-fall-2022}{%
\section{Codes and Encryption (MA/CS-124) Westmont College, Fall
2022}\label{codes-and-encryption-macs-124-westmont-college-fall-2022}}

\hypertarget{what-is-this-course-about}{%
\subsection{What is this course
about?}\label{what-is-this-course-about}}

Modern applications of computing demand that the storage and
transmission of data be secure and reliable. Cryptography is the study
of techniques for protecting data from adversaries, while coding theory
deals with representing data robustly in digital form. This course
provides an introduction to these related fields. Topics include basic
number theory and modern algebra, classical and modern cryptosystems,
discrete logarithms, hash functions, digital signatures, elliptic
curves, and error-correcting codes.

We will approach these topics from a mathematical, yet applied,
perspective. We will focus on the general mathematical principles that
govern the ways that information is stored and transmitted securely. We
will not dwell too heavily on the technicalities of specific protocols
and implementations. Rather, we strive to understand the concepts that
underlie the technology, so that you will be prepared to view current
and future implementations as scientists and engineers, not as
technicians.

\hypertarget{is-there-a-textbook}{%
\subsection{Is there a textbook?}\label{is-there-a-textbook}}

We will cover most of the content of \emph{Introduction to Cryptography
with Coding Theory}, by Trappe and Washington, 2nd Edition. If you have
your own computer, install R and RStudio on it
(\url{https://www.rstudio.com}). We will be using Canvas, so you might
want to install the app. It will also be convienent to have an app on
your phone that can scan documents to PDF.

\hypertarget{what-is-the-coursework-and-how-is-it-graded}{%
\subsection{What is the coursework and how is it
graded?}\label{what-is-the-coursework-and-how-is-it-graded}}

You should expect a \textbf{written assignment} due the night before
every class meeting. You will submit them on Canvas as PDFs, which you
can create by scanning your written work using a scanning app. If you
prefer, you can typeset these assignments using LaTeX or a word
processor. Over the course of the semester, you will develop an
\textbf{R package} containing a library of functions for solving
problems in cryptography and coding theory. Typically, I will give you a
template including function prototypes, ROxygen comments, and test cases
that your functions should satisfy. You will submit successive versions
of this package as \texttt{tar.gz} files on Canvas. There will be two
written midterm \textbf{exams}, and a cumulative \textbf{final exam} on
the scheduled date. The following table shows how these assessments are
weighted to determine your final grade.

\begin{tabular}[t]{ll}
\toprule
Written Assignments & 25\%\\
R Course Package & 15\%\\
Midterms & 20\% each\\
Final Exam & 20\%\\
\bottomrule
\end{tabular}

\hypertarget{what-other-policies-should-students-be-aware-of}{%
\subsection{What other policies should students be aware
of?}\label{what-other-policies-should-students-be-aware-of}}

If you miss a significant number of classes, you will almost definitely
do poorly in this class. If you miss more than four classes without a
valid excuse, I reserve the right to terminate you from the course with
a failing grade. Work missed (including tests) without a valid excuse
will receive a zero.

I expect you to check your email on a regular basis. If you use a
non-Westmont email account, please forward your Westmont email to your
preferred account. I'll send out notices on Canvas, so make sure you
receive Canvas notifications in your email.

Learning communities function best when students have academic
integrity. Cheating is primarily an offense against your classmates
because it undermines our learning community. Therefore, dishonesty of
any kind may result in loss of credit for the work involved and the
filing of a report with the Provost's Office. Major or repeated
infractions may result in dismissal from the course with a failing
grade. Be familiar with the College's plagiarism policy, found at
\url{https://www.westmont.edu/office-provost/academic-program/academic-integrity-policy}.

In particular, providing someone with an electronic copy of your work is
a breach of the academic integrity policy. Do not email, post online, or
otherwise disseminate any of the work that you do in this class. If you
keep your work on a repository, make sure it is private. You may work
with others on the assignments, but make sure that you write or type up
your own answers yourself. You are on your honor that the work you hand
in represents your own understanding.

\hypertarget{other-information}{%
\subsection{Other Information}\label{other-information}}

\begin{description} 

\item[Professor:] David J. Hunter, Ph.D.
  (\verb!dhunter@westmont.edu!). Student hours are in Winter Hall 303 
  from 1:30--4pm on Tuesdays and Thursdays.

 \item[Tentative Schedule:] The following schedule is a rough first approximation of the topics in \textit{Trappe} that we plan to cover; it is subject to revision at the instructor's discretion. Chapter 3 (Basic Number Theory) will be covered throughout the course when the relevant topics arise.
  \begin{itemize}
      \item Chapter 2: Classical Cryptosystems
      \item Chapters 4--5: DES and AES
      \item Chapter 6: The RSA Algorithm
   \begin{quote}
    \textit{Midterm \#1}     (through Chapter 6)
   \end{quote}
      \item Chapter 7: Discrete Logarithms
      \item Chapter 8: Hash Functions
      \item Chapter 9: Digital Signatures
      \item Chapter 16: Elliptic Curves
   \begin{quote}
    \textit{Midterm \#2}     (through Chapter 16)
   \end{quote}
      \item Chapter 18: Error Correcting Codes
   \begin{quote}
    \textit{Final Exam}     (cumulative, with an emphasis on Chapter 18)
   \end{quote}
  \end{itemize}

\item[Accommodations for Students with Disabilities:] Students who have been diagnosed with a disability (learning, physical or psychological) are strongly encouraged to contact the Disability Services office as early as possible to discuss appropriate accommodations for this course. Formal accommodations will only be granted for students whose disabilities have been verified by the Disability Services office.  These accommodations may be necessary to ensure your equal access to this course.  Please contact Sheri Noble, Director of Disability Services (310A Voskuyl Library, 565-6186, \href{mailto:snoble@westmont.edu}{\tt snoble@westmont.edu}) or visit \url{https://www.westmont.edu/disability-services} for more information.

\item[Program and Institutional Learning Outcomes:] The
         mathematics department at Westmont College has formulated the
         following learning outcomes for all of its classes. (PLO's)
\begin{enumerate}[noitemsep]
\item Core Knowledge: Students will demonstrate knowledge of the
                  main concepts, skills, and facts of the discipline of
                  mathematics.
\item Communication: Students will be able to communicate mathematical ideas
     following the standard conventions of writing or speaking in the
     discipline.
\item Creativity: Students will demonstrate the ability to formulate and make
     progress toward solving non-routine problems.
\item Christian Connection: Students will incorporate their mathematical skills
     and knowledge into their thinking about their vocations as followers of
     Christ.
         \end{enumerate}
         In addition, the faculty of Westmont College have established common
         learning outcomes for all courses at the institution
         (ILO's). These outcomes are summarized as follows:
(1) Christian Understanding, Practices, and Affections,
(2) Global Awareness and Diversity,
(3) Critical Thinking,
(4) Quantitative Literacy,
(5) Written Communication,
(6) Oral Communication, and
(7) Information Literacy.

\item[Course Learning Outcomes:] The above outcomes are reflected in the
     particular learning outcomes for this course.
     After taking this course, you should be able
     to:
    \begin{itemize}
        \item Demonstrate understanding of the theoretical basis for cryptography and coding theory.
             (PLO 1, ILOs 3,4)
        \item Write and evaluate mathematical arguments according to the
             standards of the discipline. (PLO 2,
              ILOs 3,5)
        \item Construct solutions to novel problems,
               demonstrating perseverance in the face of open-ended or
               partially-defined contexts. (PLO 3, ILO 3)
        \item Consider the ethical implications of the subject matter. (PLO 4, ILO 1)
    \end{itemize}
These outcomes will be assessed by written assignments, programming assignments, and written exams, as described above.

\end{description}

\end{document}
